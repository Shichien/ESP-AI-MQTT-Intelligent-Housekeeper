
\subsection{项目布局及构建}

\begin{tikzpicture}
    % 绘制 forest 树结构
    \node at (0, 0) {
        \begin{forest}
            for tree={
                grow=east,
                draw,
                edge={-latex},
                rounded corners,
                node options={align=center},
                anchor=west,
                parent anchor=east,
                child anchor=west,
                delay={where content={}{shape=coordinate}{}},
            }
            [ESP32
                [Master
                    [/src
                        [main.cpp]
                        [Config]
                        [Authorization]
                        [STT]
                        [LLM]
                        [TTS]
                        [CommunicationModule
                            [ESPNOW]
                            [Network]
                        ]
                        [SpeakToUser]
                        [Audio]
                    ]
                    [/.pio]
                    [platfromio.ini]
                ]
                [Slave
                    [/src
                        [DHT22]
                        [EspNow]
                        [MG90Servo]
                        [MQ]
                        [MQTT]
                        [OtherSensors]
                        [Snake]
                        [configuration]
                        [main.cpp]
                    ]
                    [.pio]
                    [platfromio.ini]
                ]
            ]
        \end{forest}
    };

    % 在右侧添加文本
    \node[anchor=west] at (4, 0) { % 调整 x 方向的偏移量以调整文本位置
        \begin{minipage}{0.4\textwidth} % 调整 minipage 的宽度来控制文本框大小
            \textbf{Project Structure:}\\
            项目的树状结构如图所示,最初编译时我将所有的头文件都置于Configuration中,每一份文件都包含它,导致编译非常缓慢,于是将不重合的部分库取出,单独置于需要的文件中,编译速度显著提升。
        \end{minipage}
    };

\end{tikzpicture}

\subsection{参考资料}
\begin{itemize}
    \item Ollama API文档:\href{https://github.com/ollama/ollama/blob/main/docs/api.md\#create-a-model}{\underline{https://github.com/ollama/ollama/blob/main/docs/api.md\#create-a-model}}
    \item 心知天气API文档:\href{https://seniverse.yuque.com/hyper\_data/api\_v3}{\underline{https://seniverse.yuque.com/hyper\_data/api\_v3}}
    \item 讯飞API接口文档:\href{https://xfyun.cn/doc}{\underline{https://xfyun.cn/doc}}
    \item OneNET平台文档:\href{https://open.iot.10086.cn/doc/v5/fuse/detail/920}{\underline{https://open.iot.10086.cn/doc/v5/fuse/detail/920}}
    \item —————————— 参考了很多Github库中的Example —————————————
    \item ESPRessif各种库文档:\href{https://docs.espressif.com/projects/arduino-esp32/en/latest/libraries.html}{\underline{https://docs.espressif.com/projects/arduino-esp32/en/latest/libraries.html}}
    \item U8g2 OLED库:\href{https://github.com/olikraus/u8g2}{\underline{https://github.com/olikraus/u8g2}}
    \item ArduinoJson库:\href{https://arduinojson.org/}{\underline{https://arduinojson.org/}}
    \item HTTPClient库:\href{https://github.com/espressif/arduino-esp32/tree/master/libraries/HTTPClient}{\underline{https://github.com/espressif/arduino-esp32/tree/master/libraries/HTTPClient}}
    \item ESP-I2S协议库:\href{https://github.com/schreibfaul1/ESP32-audioI2S}{\underline{https://github.com/schreibfaul1/ESP32-audioI2S}}
    \item Adafruit开源硬件公司各常用模块、传感器介绍网站:\href{https://lastminuteengineers.com/electronics/arduino-projects/}{\underline{https://lastminuteengineers.com/electronics/arduino-projects/}}
    \item ——————————————————————————————————————————
    \item 【波特律动】在线串口调试助手:\href{https://serial.keysking.com/}{\underline{https://serial.keysking.com/}}
    \item VOFA+ 1.3.10:\href{https://www.vofa.com/}{\underline{https://www.vofa.com/}}
    \item ——————————————————————————————————————————
    \item MPU6050姿态解算2-欧拉角\&旋转矩阵:\href{https://zhuanlan.zhihu.com/p/195683958}{\underline{https://zhuanlan.zhihu.com/p/195683958}}
    \item 学习心得|基于卡尔曼滤波的MPU6050姿态解算:\href{https://www.bilibili.com/video/BV1sL411F7fu}{\underline{https://www.bilibili.com/video/BV1sL411F7fu}}
    \item 对MQ系列传感器采集电压与浓度转换的公式的探索:\href{https://zhuanlan.zhihu.com/p/453499554}{\underline{https://zhuanlan.zhihu.com/p/453499554}}
    \item 南京大学CSI前沿技术分享:\href{https://www.bilibili.com/video/BV1Cv411q7Cz}{\underline{https://www.bilibili.com/video/BV1Cv411q7Cz}}
    \item 新版微信小程序连接到OneNET平台:\href{https://blog.csdn.net/2401\_83704192/article/details/138913230}{https://blog.csdn.net/2401\_83704192/article/details/138913230}
    \item ESP32发声:\href{https://github.com/MetaWu2077/Esp32\_VoiceChat\_LLMs}{https://github.com/MetaWu2077/Esp32\_VoiceChat\_LLMs}
    \item 微信开发者工具使用文档:\href{https://developers.weixin.qq.com/miniprogram/dev/devtools/edit.html}{https://developers.weixin.qq.com/miniprogram/dev/devtools/edit.html}
    \item ——————————————————————————————————————————
    \item Unix时间戳转换工具:\href{https://www.jyshare.com/front-end/852/}{\underline{https://www.jyshare.com/front-end/852/}}
    \item ESPNOW 通信不成功:\href{https://esp32.com/viewtopic.php?p=132946}{\underline{https://esp32.com/viewtopic.php?p=132946}}
    \item ——————————————————————————————————————————
    \item 华东师范大学软件学院实验报告、Beamer模板(本人所作):\href{https://github.com/Shichien/ECNU-LateX-Template}{https://github.com/Shichien/ECNU-LateX-Template}
\end{itemize}

\subsection*{Special Thanks}
 @ Azazo1

 @ Luryem