
\subsection{项目介绍}

本项目以 ESP32 为主控,使用 LLM(讯飞星火大模型)运行语音识别与在线联网回答。
基于 MQTT、HTTP、ESP-NOW 协议实现远程云端传输、近程数据传输。
通过外接传感器实现动态数据读取,在线数据反馈,实现了基于物联网的智能外接设备控制系统。

课程项目源代码及相关文件链接:\href{https://github.com/Shichien/Maker-practice}{\underline{https://github.com/Shichien/Maker-practice}}

\subsection{项目主要功能及应用场景}

本项目通过多种传感器测量室内示数:如$CH_4、H_2、CH_3COOH、LPG、温度、湿度$等,
实时通过 MQTT 协议传输至云端,在各地均可通过微信小程序或 OneNET 平台调试器进行远程读数,获取室内情况,以及远程控制室内各模块。
项目同时实现全智能无线连接,通过语音识别判断用户需求,实时语音播报室内信息;连接大模型,模块可联网实现提问与回答。

通过 ESP-NOW 协议绕过 WiFi ,实现伪 AC + AP 组网模式,多接入点可实现任意规模大小接入,通过 ESP-NOW 协议实现短距离无线通信,
可以连接任意接口进行函数调用输出。

火焰传感器与光照传感器、雨滴传感器可实时监控室内的特殊天气、事件情况,传输给云端进行数据处理,报警等操作。

\subsection{硬件物料}

\begin{itemize}
    \item ESP-WROOM-32 开发板、ESP-8266-Mod 开发板
    \item 1.8 寸 RGB-TFT OLED、SSD 1306 OLED
    \item INMP441 MEMS 麦克风
    \item MAX98357 I2S 音频放大器
    \item LB 喇叭
    \item DHT22、MPU6050、LM393、MQ5、MQ135 传感器
    \item MH-RD Raindrops Module
    \item 无刷电机(Fan Module)
\end{itemize}

    具体的接线图及引脚定义图请见【\textcolor{mygreen}{附件}】

\subsection{To Do List}

\begin{itemize}
    \item 上传时删除库文件中的 Sample 例子
    \item 使用 Ollama 部署本地推断模型与 ESP32 交汇
    \item 基于阿里云进行 MQTT 协议通信以及远程控制、点灯等操作
    \item 在 PlatformIO 平台上有一个库是 Aliyun IOT SDK,注意后续可能要用到
    \item 使用 MPU6050 进行云台自控制
\end{itemize}

\subsection{项目环境及工具列表}

本项目设计的环境为 Windows 11,开发板编程使用 Clion Platform,
选择 Upesy\_Wroom 开发环境,Arduino 框架,模拟仿真平台使用 Wokwi Simulator,
串口监视器选用 VOFA+ 1.3.10,波特率为115200(8N1),代码使用 Vim 编写,
课程报告使用 \LaTeX 撰写,并使用Git进行版本管理。

% —————————————————————————————————————————————————————————————

\section{技术原理}

\subsection{ESP32的使用及其硬件接口}

\begin{itemize}
    \item Espressif 32ESP32 集成了 2.4 GHz Wi-Fi 和 Bluetooth 4.2 LE(低功耗)功能,使其能够在无线网络环境中轻松实现连接和数据传输。
    \item 提供了丰富的外设接口,包括多达 34 个 GPIO 引脚、ADC(模数转换器)、DAC(数模转换器)、SPI、I2C、UART、PWM、CAN 总线。
\end{itemize}

\subsection{基于LLM、TTS语音识别与合成}

随着 LLM 在近年来在世界各地火热传播,一位机器学习工程师与我说:“现在市场上是个软件就得说自己用了 AI,不然不好意思提供服务了”,此句稍带滑稽的话语让我萌生了使用LLM的念头。
灵活高效、易于使用。在本项目中,主要用于流式语音识别及语音生成转换(基于讯飞的星火大模型Spark Ultra 4.0)。

\subsection{FreeRTOS、Ticker多任务调度}

FreeRTOS 是一个广泛用于嵌入式开发的实时操作系统,提供多任务调度、任务间通信、优先级管理等功能,
vTaskDelay() 是 FreeRTOS 提供的任务延迟函数,用于让当前任务进入阻塞状态,释放 CPU 给其他任务使用。
ESP32已经包含了FreeROTS内核,使用 vTaskDelay,可以避免Delay()函数造成的整体阻塞。

同时,Ticker计时器可以使用一行代码实现多任务调度的管理:\texttt{ticker.attach(Time,FunctionName);}

\subsection{基于HTTP、ESP-NOW通信协议的使用}

ESP-NOW 是一种轻量级、低功耗的点对点(P2P)通信协议,专为短距离设备之间的快速数据传输而设计。
不依赖于传统的 Wi-Fi 网络和路由器。
ESP-NOW 允许多个设备(资料显示,最多20个,不然会超过ESP32的功耗)直接进行无线数据传输,适用于需要快速和实时通信的物联网应用场景。

HTTP 在客户端与服务器之间传输超文本数据,是互联网通信的基础,HTTPClient库提供各种接口,包括GET、POST请求,可以连接服务器后返回响应:
HTML、JSON数据等,使用ESP32可以远程连接各种API获取数据,如使用\underline{www.baidu.com} 获取 RFC1123 时间戳。

\subsection{基于MQTT物联网通信协议的使用}

MQTT 是一个轻量级的发布/订阅消息传输协议,广泛用于物联网设备之间的数据通信。
提供在不稳定网络环境中的低带宽、高延迟或间歇性连接情况下的可靠通信。
客户端可以是任何设备或应用程序(如传感器、服务器、移动设备),
它们可以作为发布者,也可以作为订阅者接收消息。

OneNET 是中国移动开发的物联网(IoT)开放平台,支持多种主流通信协议,如MQTT、HTTP、CoAP等,在此我使用MQTT协议进行传感器的上传。

\subsection{多种传感器综合实践}

Adafruit开源硬件公司各常用模块以及MH-Sensor系列是一系列传感器产品,包括温度传感器、湿度传感器、气压传感器
光照传感器等。这些传感器可以用于测量环境参数,例如温度、湿度、气压、光照等。

\subsubsection{MPU6050 六轴传感器}

MPU6050 集成了三轴加速度计和三轴陀螺仪,能够测量 X、Y、Z 三个方向上的加速度和角速度。
加速度计的测量范围为 ±2g、±4g、±8g 和 ±16g。
陀螺仪的测量范围为 ±250°/s、±500°/s、±1000°/s 和 ±2000°/s。
采用IIC总线接口通信。

\subsubsection{DHT22 湿温度传感器}

温度测量范围: -40°C 到 80°C,精度为 ±0.5°C。
湿度测量范围: 0\% 到 100\% 相对湿度,精度为 ±2\% RH。

DHT22 使用单线数字信号输出,易于与微控制器进行接口通信。

\subsubsection{LM393光照传感器}

LM393 的工作原理基于电压比较。
当正输入电压高于负输入电压时,输出为低电平(接近 0V)。
当正输入电压低于负输入电压时,输出为高电平(受上拉电阻决定),内置光敏电阻,可提供模拟信号。

\subsubsection{MQ系列气体传感器}

MQ传感器内部通常由一个加热元件和一个氧化锡(SnO2)半导体组成。当特定气体通过传感器时,氧化锡半导体的电导率会发生变化,这种变化可以被检测并转化为气体浓度。
每种MQ传感器的灵敏度曲线不同,对某些气体更敏感。例如,MQ-2对甲烷、丙烷和烟雾敏感,而MQ-7对一氧化碳更敏感。

注意:\textbf{MQ系列传感器一次只能检测一种气体的浓度值,但对于不同的气体分子,拥有不同的指数函数曲线,在定义不同的A、B值时读取}

\subsubsection{火焰传感器}

火焰传感器可以用来探测火源或其它波长在760纳米~1100纳米范围内的光源。
探测角度达60度,对火焰光谱特别灵敏。
注意,手机手电筒中也含有对应波长的光源。